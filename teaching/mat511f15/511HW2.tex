\documentclass[11pt]{scrartcl}
\usepackage[scale=1.5]{ccicons}
\usepackage[notextcomp]{kpfonts} 
\usepackage[margin=1in]{geometry}
\usepackage{amsthm,amssymb}
\usepackage{graphicx}
\usepackage{enumitem}
\usepackage{bm}
\usepackage{tabu}

\usepackage{color}
\definecolor{darkblue}{rgb}{0, 0, .6}
\definecolor{grey}{rgb}{.7, .7, .7}
\usepackage[breaklinks]{hyperref}
\hypersetup{
	colorlinks=true,
	linkcolor=darkblue,
	anchorcolor=darkblue,
	citecolor=darkblue,
	pagecolor=darkblue,
	urlcolor=darkblue,
	pdftitle={},
	pdfauthor={}
}

\usepackage{fancyhdr}
\pagestyle{fancy}
\lhead{MAT 511 - Fall 2015}
\chead{}
\rhead{Due Wednesday, September 9}
%\lfoot{}%\scriptsize This work is licensed under the \href{http://creativecommons.org/licenses/by-sa/3.0/us/}{Creative Commons Attribution-Share Alike 3.0 License}.} 
%\cfoot{}
%\rfoot{\ccbysa}
\renewcommand{\headrulewidth}{.4pt}
%\renewcommand{\footrulewidth}{.4pt}

\theoremstyle{definition}
\newtheorem{theorem}{Theorem}
\newtheorem{acknowledgement}[theorem]{Acknowledgement}
\newtheorem{algorithm}[theorem]{Algorithm}
\newtheorem{axiom}[theorem]{Axiom}
\newtheorem{case}[theorem]{Case}
\newtheorem{claim}[theorem]{Claim}
\newtheorem*{claim*}{Claim}
\newtheorem{conclusion}[theorem]{Conclusion}
\newtheorem{condition}[theorem]{Condition}
\newtheorem{conjecture}[theorem]{Conjecture}
\newtheorem{corollary}[theorem]{Corollary}
\newtheorem{criterion}[theorem]{Criterion}
\newtheorem{definition}[theorem]{Definition}
\newtheorem{example}[theorem]{Example}
\newtheorem{exercise}[theorem]{Exercise}
\newtheorem{journal}[theorem]{Journal}
\newtheorem{lemma}[theorem]{Lemma}
\newtheorem{notation}[theorem]{Notation}
\newtheorem{problem}[theorem]{Problem}
\newtheorem{proposition}[theorem]{Proposition}
\newtheorem{remark}[theorem]{Remark}
\newtheorem{solution}[theorem]{Solution}
\newtheorem{summary}[theorem]{Summary}
\newtheorem{skeleton}[theorem]{Skeleton Proof}
\newtheorem{activity}[theorem]{Activity}
\newtheorem{intuitivedef}[theorem]{Intuitive Definition}

\newcommand{\blankline}{\pagebreak[2]\vspace{.5\baselineskip}}

\setlength{\parindent}{0pt}

%Useful for cut and paste
%\begin{enumerate}[label=\rm{(\alph*)}]

\begin{document}

\title{Homework 2}
\subtitle{Abstract Algebra I}
\date{}

\maketitle
\thispagestyle{fancy}

Complete the following problems.

\begin{problem}
Determine whether each of the following binary operations is (i) associative and (ii) commutative.
\begin{enumerate}[label=\rm{(\alph*)}]
\item The operation $\star$ on $\mathbb{R}$ defined via $a\star b=a+b+ab$.
\item The operation $\circ$ on $\mathbb{Q}$ defined via $\displaystyle a\circ b=\frac{a+b}{5}$.
\item The operation $\circledcirc$ on $\mathbb{Z}\times \mathbb{Z}$ defined via $(a,b)\circledcirc (c,d)=(ad+bc,bd)$.
\item The operation $\circledast$ on $\mathbb{Q}\setminus\{0\}$ defined via $\displaystyle a\circledast b=\frac{a}{b}$.
\item The operation $\circleddash$ on $\mathbb{R}/I:=\{x\in\mathbb{R}\mid 0\leq x<1\}$ defined via $a\circleddash b=a+b-\lfloor a+b\rfloor$ (i.e., $a\circleddash b$ is the fractional part of $a+b$).
\end{enumerate}
\end{problem}

\begin{problem}
Determine which of the following sets are groups under the given operation. Justify your answer.
\begin{enumerate}[label=\rm{(\alph*)}]
\item $\mathbb{Z}/n\mathbb{Z}$ under multiplication mod $n$.
\item Set of rational numbers in lowest terms whose denominators are odd under addition.
\item Set of rational numbers in lowest terms whose denominators are even together with $0$ under addition.
\item Set of rational numbers of absolute value less than 1 under addition.
\item $\mathbb{R}/I$ under $\circleddash$ as defined in Problem 1(e). 
\end{enumerate}
\end{problem}

\begin{problem}
Let $G=\{a+b\sqrt{2}\mid a,b\in\mathbb{Q}\}$.  Prove one of the following.
\begin{enumerate}[label=\rm{(\alph*)}]
\item The set $G$ is a group under addition.
\item If $H=G\setminus\{0\}$, then $H$ is a group under multiplication.
\end{enumerate}
\end{problem}

\begin{problem}
Assume $G$ is a group and let $x\in G$.  Prove one of the following.
\begin{enumerate}[label=\rm{(\alph*)}]
\item If $a,b\in\mathbb{Z}$, then $x^{a+b}=x^ax^b$.
\item If $a,b\in\mathbb{Z}$, then $(x^a)^b=x^{ab}$.
\end{enumerate}
Don't forget to handle the case when either $a$ or $b$ is nonpositive.
\end{problem}

\begin{problem}
Assume $G$ is a group and let $a,b\in G$.  Is it true that $(ab)^n=a^nb^n$?  If not, under what minimal conditions would it be true? Prove the statement that you think is true.
\end{problem}

\begin{problem}
Assume $G$ is a group. Prove that if $x^2=e$ for all $x\in G$, then $G$ is abelian.
\end{problem}

\begin{problem}
Assume $(G,\star)$ is a group and let $H$ be a nonempty subset of $G$ that is (i) closed under $\star$ and (ii) closed under inverses (i.e., for all $h,k\in H$, (i) $hk\in H$ and (ii) $h^{-1}\in H$).  Prove that $H$ is a group under $\star$ in its own right.  Such a subset is called a \emph{subgroup}.
\end{problem}

\begin{problem}
Assume $G$ is a group.  Prove that if $x\in G$ such that $x^n\neq e$ for all $n\in \mathbb{Z}^+$, then $x^i\neq x^j$ for all $i\neq j$.
\end{problem}

\begin{problem}
Assume $G=\{e,a,b,c\}$ is a group under $\star$ with the property that $x^2=x^4$ for all $x\in G$ (where $e$ is the identity). Complete the following \emph{group table}, where $x\star y$ is defined to be the entry in the row labeled by $x$ and the column labeled by $y$.

\begin{center}
\begin{tabu}{c|[2pt]c|c|c|c}
$\star$ & $e$ & $a$ & $b$ & $c$ \\ \tabucline[2pt]{-}
$e$ &  $e$ & $a$ & $b$ & $c$ \\
\hline $a$ & $a$ &  & & \\
\hline $b$ & $b$ & & & \\
\hline $c$ & $c$ & & &
\end{tabu}
\end{center}
Is your table unique?  That is, did you have to fill it out the way you did?  Deduce that $G$ is abelian.
\end{problem}

\begin{problem}
\textbf{(Optional)} Assume $G$ is a finite group.  Prove that every element of $G$ must appear exactly once in every row and column of the group table for $G$.  (Of course, we are not counting the row and column headings.)
\end{problem}

\end{document}