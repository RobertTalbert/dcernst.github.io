\documentclass[11pt]{scrartcl}
\usepackage[scale=1.5]{ccicons}
\usepackage[notextcomp]{kpfonts} 
\usepackage[margin=1in]{geometry}
\usepackage{amsthm,amssymb}
\usepackage{graphicx}
\usepackage{enumitem}
\usepackage{bm}
\usepackage{tabu}
\usepackage{tikz}

\usepackage{color}
\definecolor{darkblue}{rgb}{0, 0, .6}
\definecolor{grey}{rgb}{.7, .7, .7}
\usepackage[breaklinks]{hyperref}
\hypersetup{
	colorlinks=true,
	linkcolor=darkblue,
	anchorcolor=darkblue,
	citecolor=darkblue,
	pagecolor=darkblue,
	urlcolor=darkblue,
	pdftitle={},
	pdfauthor={}
}

\usepackage{fancyhdr}
\pagestyle{fancy}
\lhead{MAT 612 - Spring 2016}
\chead{}
\rhead{Due Wednesday, February 3}
\renewcommand{\headrulewidth}{.4pt}

\theoremstyle{definition}
\newtheorem{theorem}{Theorem}
\newtheorem{acknowledgement}[theorem]{Acknowledgement}
\newtheorem{algorithm}[theorem]{Algorithm}
\newtheorem{axiom}[theorem]{Axiom}
\newtheorem{case}[theorem]{Case}
\newtheorem{claim}[theorem]{Claim}
\newtheorem*{claim*}{Claim}
\newtheorem{conclusion}[theorem]{Conclusion}
\newtheorem{condition}[theorem]{Condition}
\newtheorem{conjecture}[theorem]{Conjecture}
\newtheorem{corollary}[theorem]{Corollary}
\newtheorem{criterion}[theorem]{Criterion}
\newtheorem{definition}[theorem]{Definition}
\newtheorem{example}[theorem]{Example}
\newtheorem{exercise}[theorem]{Exercise}
\newtheorem{journal}[theorem]{Journal}
\newtheorem{lemma}[theorem]{Lemma}
\newtheorem{notation}[theorem]{Notation}
\newtheorem{problem}[theorem]{Problem}
\newtheorem{proposition}[theorem]{Proposition}
\newtheorem{remark}[theorem]{Remark}
\newtheorem{solution}[theorem]{Solution}
\newtheorem{summary}[theorem]{Summary}
\newtheorem{skeleton}[theorem]{Skeleton Proof}
\newtheorem{activity}[theorem]{Activity}
\newtheorem{intuitivedef}[theorem]{Intuitive Definition}

\DeclareMathOperator{\Aut}{Aut}
\DeclareMathOperator{\Inn}{Inn}
\DeclareMathOperator{\Stab}{Stab}

\newcommand{\blankline}{\pagebreak[2]\vspace{.5\baselineskip}}

\setlength{\parindent}{0pt}

%Useful for cut and paste
%\begin{enumerate}[label=\rm{(\alph*)}]

\begin{document}

\title{Homework 2}
\subtitle{Abstract Algebra II}
\date{}

\maketitle
\thispagestyle{fancy}

Complete the following problems. Note that you should only use results that we've discussed so far this semester or last semester.

\begin{problem}
Let $R$ be a Euclidean Domain with norm $N$ satisfying $N(a)\leq N(ab)$ for all nonzero $a,b\in R$\footnote{This extra requirement on $N$ is sometimes part of the definition of Euclidean Domain.}.  Prove that $a\in R$ is a unit iff $N(a)=N(1)$. \emph{Note:} Actually, I don't think we need this extra requirement on $N$, but since I already modified the problem, I'll just leave it.
\end{problem}

\begin{problem}
Consider the Euclidean Domain $\mathbb{Z}[i]$ with norm given by $N(a+bi)=a^2+b^2$.  
\begin{enumerate}[label=\rm{(\alph*)}]
\item Find the units in $\mathbb{Z}[i]$.
\item For each of the following pairs, find $q$ and $r$ such that $a=bq+r$ with $r=0$ or $N(r)<N(b)$.
\begin{enumerate}[label=\rm{(\roman*)}]
\item $a=11+8i,b=1+2i$
\item $a=-17+15i,b=3+i$
\end{enumerate}
\end{enumerate}
\end{problem}

\begin{problem}
Consider the Euclidean Domain $\mathbb{Q}[x]$ with norm given by $N(p(x))=\deg(p(x))$. 
\begin{enumerate}[label=\rm{(\alph*)}]
\item Prove that $(x^2+1,x^3+1)=\mathbb{Q}[x]$.
\item Find polynomials $a(x)$ and $b(x)$ such that $(x^2+1)a(x)+(x^3+1)b(x)=1$.
\end{enumerate}
\end{problem}

\begin{problem}
Let $R$ be an integral domain and let $u$ be a unit of $R$.
\begin{enumerate}[label=\rm{(\alph*)}]
\item Prove that if $p\in R$ is prime, then $up$ is prime.
\item Prove that if $p\in R$ is irreducible, then $up$ is irreducible.
\end{enumerate}
\end{problem}

\begin{problem}
Consider the ring $\mathbb{Z}[\sqrt{-5}]$ with norm $N(a+b\sqrt{-5})=a^2+5b^2$.  
\begin{enumerate}[label=\rm{(\alph*)}]
\item Justify my claim in Example~1.85(3) that $6=2\cdot 3=(1+\sqrt{-5})(1-\sqrt{-5})$ are two distinct factorizations of 6 into irreducibles in $\mathbb{Z}[\sqrt{-5}]$. \emph{Hint:} Start by showing that $N(xy)=N(x)N(y)$ for all $x,y\in \mathbb{Z}[\sqrt{-5}]$.
\item Prove that $1+\sqrt{-5}$ is not prime in $\mathbb{Z}[\sqrt{-5}]$.
\end{enumerate}
\end{problem}

\begin{problem}
Consider the ring $\mathbb{Z}[2\sqrt{2}]$.
\begin{enumerate}[label=\rm{(\alph*)}]
\item Prove that $\mathbb{Z}[2\sqrt{2}]$ is not a UFD. \emph{Hint:} Fiddle around with 8.  You will need to justify that certain ring elements are irreducibles.  One way to do this is to play with the map $N:\mathbb{Z}[2\sqrt{2}]\to \mathbb{Z}\cup\{0\}$ given by $N(a+2b\sqrt{2})=|a^2-8b^2|$ (the vertical bars denote absolute value). It would be useful to know $N(rs)=N(r)N(s)$, $N(r)=1$ iff $r$ is a unit, and $N(r)\neq 2$ for all $r\in R$. If you want to use these facts, you should prove them.
\item If possible, give an example of an ideal of $\mathbb{Z}[2\sqrt{2}]$ that is not principal.  If not possible, briefly explain why.
\end{enumerate}
\end{problem}

\end{document}