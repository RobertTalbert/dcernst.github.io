\documentclass[11pt]{scrartcl}

\usepackage[scale=1.5]{ccicons}
\usepackage[notextcomp]{kpfonts}
\usepackage{multicol}
\usepackage{url}
\usepackage{array}
\usepackage{multicol}
\usepackage{tabu}
\usepackage{tikz}
\usetikzlibrary{shapes.geometric}
\usepackage{fancyhdr}
\usepackage[margin=1in]{geometry}
\usepackage[hang,flushmargin,symbol*]{footmisc}
\usepackage{amsmath}
\usepackage{amsthm}
\usepackage{amssymb}
\usepackage{mathtools}
\usepackage{enumitem}
\usepackage{graphicx}
\usepackage{color}
\definecolor{darkblue}{rgb}{0, 0, .6}
\definecolor{grey}{rgb}{.7, .7, .7}
\usepackage[breaklinks]{hyperref}

\theoremstyle{definition} 
\newtheorem{theorem}{Theorem}
\newtheorem{lemma}[theorem]{Lemma}
\newtheorem{claim}[theorem]{Claim}
\newtheorem{corollary}[theorem]{Corollary}
\newtheorem{conjecture}[theorem]{Conjecture}
\newtheorem{definition}[theorem]{Definition}
\newtheorem{example}[theorem]{Example}
\newtheorem{remark}[theorem]{Remark}
\newtheorem{important}[theorem]{Important Note}
\newtheorem{recall}[theorem]{Recall}
\newtheorem{note}[theorem]{Note}
\newtheorem{question}[theorem]{Question}
\newtheorem*{definition*}{Definition}

\newcommand{\ds}{\displaystyle}
\newcommand{\lcm}{\operatorname{lcm}}
\newcommand{\Rng}{\operatorname{Rng}}
\DeclareMathOperator{\Aut}{Aut}
\DeclareMathOperator{\Gal}{Gal}
\DeclareMathOperator{\Tor}{Tor}
\DeclareMathOperator{\Hom}{Hom}

\setlength{\parindent}{0pt}
\setlength{\fboxsep}{10pt}

%%%%%%Header/Footer%%%%%%%

\pagestyle{fancy}

\lhead{MAT 612 - Spring 2016}
\chead{}
\rhead{Final Exam (Take-home portion)}
\lfoot{\scriptsize This work is licensed under the \href{https://creativecommons.org/licenses/by-sa/4.0/}{Creative Commons Attribution-Share Alike 4.0 License}.} 
\cfoot{}
\rfoot{\ccbysa}
\renewcommand{\headrulewidth}{.4pt}
\renewcommand{\footrulewidth}{.4pt}

%%%%%%%%%%%%%%%%%%%

\begin{document}

\begin{center}

  \fbox{\parbox{6in}{
    \vspace{5pt}
    \textbf{\large Your Name:}
    \vspace{5pt}
  }}
  
  \bigskip
  
  \fbox{\parbox{6in}{
    \vspace{5pt}
    \textbf{\large Names of Any Collaborators:}
    \vspace{5pt}
  }}

\end{center}

\section*{Instructions}

This portion of the Final Exam is worth a total of 16 points and is due by \textbf{5pm on Thursday, May 12}.  Your total combined score on the in-class portion and take-home portion is worth 20\% of your overall grade.  

\bigskip

I expect your solutions to be \emph{well-written, neat, and organized}.  Do not turn in rough drafts.  What you turn in should be the ``polished'' version of potentially several drafts.  
 
\bigskip

Feel free to type up your final version.  The \LaTeX\ source file of this exam is also available if you are interested in typing up your solutions using \LaTeX.  I'll gladly help you do this if you'd like.

\bigskip

The simple rules for the exam are:

\begin{enumerate}
\item You may freely use any theorems that we have discussed in class, but you should make it clear where you are using a previous result and which result you are using.  For example, if a sentence in your proof follows from Theorem xyz, then you should say so. 
\item Unless you prove them, you cannot use any results that we have not yet covered.
\item You are \textbf{NOT} allowed to consult external sources when working on the exam.  This includes people outside of the class, other textbooks, and online resources.
\item You are \textbf{NOT} allowed to copy someone else's work.
\item You are \textbf{NOT} allowed to let someone else copy your work.
\item You are allowed to discuss the problems with each other and critique each other's work.
\end{enumerate}

\begin{center}
\textbf{I will vigorously pursue anyone suspected of breaking these rules.}
\end{center}

\bigskip

You should \textbf{turn in this cover page} and all of the work that you have decided to submit. \textbf{Please write your solutions and proofs on your own paper.}

\bigskip

To convince me that you have read and understand the instructions, sign in the box below.

\bigskip

  \fbox{\parbox{6in}{
    \vspace{5pt}
    \textbf{\large Signature:} \hfill
    \vspace{5pt}
  }}

\bigskip

Good luck and have fun!

\newpage

Complete any \textbf{four} of following problems.  Each problem is worth 4 points. Write your solutions on your own paper and please put the problems in order.  Assume $R$ is a ring with 1 and $M$ is a left $R$-module.

\begin{enumerate}
\item Let $N_1\subseteq N_2\subseteq \cdots $ be an ascending chain of submodules of $M$. Prove that $\cup_{i=1}^{\infty}N_i$ is a submodule of $M$.
\item An element $m$ of the $R$-module $M$ is called a \emph{torsion element} if $rm=0$ for some nonzero element $r\in R$. The set of torsion elements is denoted
\[
\Tor(M)=\{m\in M\mid rm=0\text{ for some nonzero }r\in R\}.
\]
Prove that if $R$ is an integral domain, then $\Tor(M)$ is a submodule of $M$ (called the \emph{torsion} submodule of $M$.
\item Let $I$ be an ideal of $R$ and let $M'$ be the subset of elements $a\in M$ that are annihilated by some power, $I^k$, of the ideal $I$ (where the power may depend on $a$). Prove that $M'$ is a submodule of $M$.
\item Let $V=\mathbb{R}^2$ and let $T$ be the linear transformation from $V$ to $V$ which is rotation clockwise about the origin by $\pi/2$ radians.  Prove the $V$ and $0$ are the only $\mathbb{R}[x]$-submodules for $T$.
\item Let $R$ be a commutative ring. Prove that $\Hom_R(R,M)$ and $M$ are isomorphic as left $R$-modules.
\item Let $\phi:M\to N$ be an $R$-module homomorphism. Prove that $\phi(\Tor(M))\subseteq \Tor(N)$.
\item Prove that if $A$ and $B$ have the same cardinality, then the free $R$-modules $F(A)$ and $F(B)$ are isomorphic as $R$-modules.
\item Prove that if $M$ is a finitely generated $R$-module that is generated by $n$ elements, then every quotient of $M$ may be generated by $n$ (or fewer) elements, and in particular, if $M$ is cyclic, then every quotient of $M$ is cyclic.
\item An $R$-module $M$ is called \emph{irreducible} if $M\neq \emptyset$ and $0$ and $M$ are the only submodules of $M$. Prove that $M$ is irreducible iff $M\neq \emptyset$ and $M$ is a cyclic module with any nonzero element as a generator.
\end{enumerate}

\end{document}