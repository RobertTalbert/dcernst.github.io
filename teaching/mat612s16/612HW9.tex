\documentclass[11pt]{scrartcl}
\usepackage[scale=1.5]{ccicons}
\usepackage[notextcomp]{kpfonts} 
\usepackage[margin=1in]{geometry}
\usepackage{amsthm,amssymb}
\usepackage{graphicx}
\usepackage{enumitem}
\usepackage{bm}
\usepackage{tabu}
\usepackage{tikz}

\usepackage{color}
\definecolor{darkblue}{rgb}{0, 0, .6}
\definecolor{grey}{rgb}{.7, .7, .7}
\usepackage[breaklinks]{hyperref}
\hypersetup{
	colorlinks=true,
	linkcolor=darkblue,
	anchorcolor=darkblue,
	citecolor=darkblue,
	pagecolor=darkblue,
	urlcolor=darkblue,
	pdftitle={},
	pdfauthor={}
}

\usepackage{fancyhdr}
\pagestyle{fancy}
\lhead{MAT 612 - Spring 2016}
\chead{}
\rhead{Due Wednesday, April 13}
\renewcommand{\headrulewidth}{.4pt}

\theoremstyle{definition}
\newtheorem{theorem}{Theorem}
\newtheorem{acknowledgement}[theorem]{Acknowledgement}
\newtheorem{algorithm}[theorem]{Algorithm}
\newtheorem{axiom}[theorem]{Axiom}
\newtheorem{case}[theorem]{Case}
\newtheorem{claim}[theorem]{Claim}
\newtheorem*{claim*}{Claim}
\newtheorem{conclusion}[theorem]{Conclusion}
\newtheorem{condition}[theorem]{Condition}
\newtheorem{conjecture}[theorem]{Conjecture}
\newtheorem{corollary}[theorem]{Corollary}
\newtheorem{criterion}[theorem]{Criterion}
\newtheorem{definition}[theorem]{Definition}
\newtheorem{example}[theorem]{Example}
\newtheorem{exercise}[theorem]{Exercise}
\newtheorem{journal}[theorem]{Journal}
\newtheorem{lemma}[theorem]{Lemma}
\newtheorem{notation}[theorem]{Notation}
\newtheorem{problem}[theorem]{Problem}
\newtheorem{proposition}[theorem]{Proposition}
\newtheorem{remark}[theorem]{Remark}
\newtheorem{solution}[theorem]{Solution}
\newtheorem{summary}[theorem]{Summary}
\newtheorem{skeleton}[theorem]{Skeleton Proof}
\newtheorem{activity}[theorem]{Activity}
\newtheorem{intuitivedef}[theorem]{Intuitive Definition}

\DeclareMathOperator{\Aut}{Aut}
\DeclareMathOperator{\Gal}{Gal}
\DeclareMathOperator{\Inn}{Inn}
\DeclareMathOperator{\Stab}{Stab}
\DeclareMathOperator{\Char}{Char}

\newcommand{\blankline}{\pagebreak[2]\vspace{.5\baselineskip}}

\setlength{\parindent}{0pt}

%Useful for cut and paste
%\begin{enumerate}[label=\rm{(\alph*)}]

\begin{document}

\title{Homework 9}
\subtitle{Abstract Algebra II}
\date{}

\maketitle
\thispagestyle{fancy}

Complete the following problems. Note that you should only use results that we've discussed so far this semester or last semester.

\begin{problem}
Suppose $[K:F]=2$. Prove that $K$ is an algebraic extension of $F$ that is the splitting field over $F$ for a collection of polynomials in $F[x]$ (i.e., prove that $K$ is a normal extension).
\end{problem}

\begin{problem}
Determine the Galois group of $f(x)=(x^2-2)(x^2-3)(x^2-5)$ over $\mathbb{Q}$.
\end{problem}

\begin{problem}
Determine the Galois group of the splitting field over $\mathbb{Q}$ of $g(x)=x^4-14x^2+9$.
\end{problem}

\begin{problem}
Let $K=\mathbb{Q}(\sqrt[8]{2},i)$, $F_1=\mathbb{Q}(i)$, and $F_2=\mathbb{Q}(\sqrt{2})$.
\begin{enumerate}[label=\rm{(\alph*)}]
\item Prove that $\Gal(K/F_1)\cong \mathbb{Z}_8$.
\item Prove that $\Gal(K/F_2)\cong D_8$.
\end{enumerate}
\end{problem}

\begin{problem}
Let $f(x) \in \mathbb{Q}[x]$. Suppose that $z\in\mathbb{C}$ is a root of $f(x)$. 
\begin{enumerate}[label=\rm{(\alph*)}]
\item Prove that $\overline{z}$ (complex conjugate of $z$) is also a root of $f(x)$.
\item Suppose $f(x)$ has degree 3. Prove that if the Galois group of the splitting field of $f(x)$ is isomorphic to $\mathbb{Z}_3$, then $f(x)$ has only real roots.
\end{enumerate}
\end{problem}

\end{document}