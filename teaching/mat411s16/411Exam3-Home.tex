\documentclass[11pt]{article}

\usepackage{url}
\usepackage{array}
\usepackage{multicol}
\usepackage{tabu}
\usepackage[table]{xcolor}
\usepackage{tikz}
\usetikzlibrary{shapes.geometric}
\usepackage{fancyhdr}
\usepackage[margin=.7in]{geometry}
\usepackage[hang,flushmargin,symbol*]{footmisc}
\usepackage{amsmath}
\usepackage{amsthm}
\usepackage{amssymb}
\usepackage{mathtools}
\usepackage{enumitem}
\usepackage{graphicx}
\usepackage{color}
\definecolor{darkblue}{rgb}{0, 0, .6}
\definecolor{grey}{rgb}{.7, .7, .7}
\usepackage[breaklinks]{hyperref}

\theoremstyle{definition} 
\newtheorem{theorem}{Theorem}
\newtheorem{lemma}[theorem]{Lemma}
\newtheorem{problem}[theorem]{Problem}
\newtheorem{claim}[theorem]{Claim}
\newtheorem{corollary}[theorem]{Corollary}
\newtheorem{conjecture}[theorem]{Conjecture}
\newtheorem{definition}[theorem]{Definition}
\newtheorem{example}[theorem]{Example}
\newtheorem{remark}[theorem]{Remark}
\newtheorem{important}[theorem]{Important Note}
\newtheorem{recall}[theorem]{Recall}
\newtheorem{note}[theorem]{Note}
\newtheorem{question}[theorem]{Question}
\newtheorem*{definition*}{Definition}
\newtheorem*{theorem*}{Theorem}
\newtheorem*{claim*}{Claim}

\newcommand{\ds}{\displaystyle}
\newcommand{\Spin}{\operatorname{Spin}}
\newcommand{\Rng}{\operatorname{Rng}}

\setlength{\parindent}{0pt}
\setlength{\fboxsep}{10pt}

%%%%%%Header/Footer%%%%%%%

\pagestyle{fancy}

\lhead{\scriptsize  MAT 411: Introduction to Abstract Algebra - Spring 2016} 
\chead{} 
\rhead{\scriptsize Exam 3 (Take-Home Portion)} 
\lfoot{\scriptsize This work is licensed under the \href{https://creativecommons.org/licenses/by-sa/4.0/}{Creative Commons Attribution-Share Alike 4.0 License}.} 
\cfoot{} 
\rfoot{\scriptsize Written by \href{http://dcernst.github.io}{D.C. Ernst}} 
\renewcommand{\headrulewidth}{0.4pt} 
\renewcommand{\footrulewidth}{0.4pt}

%%%%%%%%%%%%%%%%%%%

\newcommand{\blankline}{\pagebreak[2]\vspace{.5\baselineskip}}
\setlength{\parindent}{0pt}

%%%%%%%%%%%%%%%%%%%

\begin{document}

\begin{center}

{\Large\bf MAT 411: Introduction to Abstract Algebra}\\
\smallskip
{\Large\bf Exam 3 (Take-Home Portion)}

\bigskip

  \fbox{\parbox{7in}{
    \vspace{10pt}
    \textbf{\large Your Name:}
    \vspace{10pt}
  }}
  
  \bigskip
  
  \fbox{\parbox{7in}{
    \vspace{10pt}
    \textbf{\large Names of Any Collaborators:}
    \vspace{10pt}
  }}

\end{center}

\section*{Instructions}

This portion of Exam 3 is worth a total of 32 points and is due at the beginning of class on \textbf{Wednesday, April 27}.  Your total combined score on the in-class portion and take-home portion is worth 15\% of your overall grade.  

\bigskip

I expect your solutions to be \emph{well-written, neat, and organized}.  Do not turn in rough drafts.  What you turn in should be the ``polished'' version of potentially several drafts.  
 
\bigskip

Feel free to type up your final version.  The \LaTeX\ source file of this exam is also available if you are interested in typing up your solutions using \LaTeX.  I'll gladly help you do this if you'd like.

\bigskip

The simple rules for the exam are:

\begin{enumerate}
\item You may freely use any theorems that we have discussed in class, but you should make it clear where you are using a previous result and which result you are using.  For example, if a sentence in your proof follows from Theorem 1.41, then you should say so.
\item Unless you prove them, you cannot use any results from the course notes that we have not yet covered.
\item You are \textbf{NOT} allowed to consult external sources when working on the exam.  This includes people outside of the class, other textbooks, and online resources.
\item You are \textbf{NOT} allowed to copy someone else's work.
\item You are \textbf{NOT} allowed to let someone else copy your work.
\item You are allowed to discuss the problems with each other and critique each other's work.
\end{enumerate}

\begin{center}
\textbf{I will vigorously pursue anyone suspected of breaking these rules.}
\end{center}

\bigskip

You should \textbf{turn in this cover page} and all of the work that you have decided to submit. \textbf{Please write your solutions and proofs on your own paper.}

\bigskip

To convince me that you have read and understand the instructions, sign in the box below.

\bigskip

  \fbox{\parbox{7in}{
    \vspace{10pt}
    \textbf{\large Signature:} \hfill
    \vspace{10pt}
  }}

\bigskip

Good luck and have fun!

\newpage

\begin{problem}
(2 points each) Consider the alternating group $A_4$.  Lagrange's Theorem tells us that the possible orders of subgroups for $A_4$ are 1, 2, 3, 4, 6, and 12.
\begin{enumerate}[label=\rm{(\alph*)}]
%\item Find examples of subgroups of $A_4$ of orders 1, 2, 3, 4, and 12.
\item Write down all of the elements of order 2 in $A_4$.
\item Argue that any subgroup of $A_4$ that contains any two elements of order 2 must contain a subgroup isomorphic to $V_4$.
\item Argue that if $A_4$ has a subgroup of order 6, that it cannot be isomorphic to $R_6$.
\item It turns out that up to isomorphism, there are only two groups of order 6, namely $S_3$ and $R_6$.  Suppose that $H$ is a subgroup of $A_4$ of order 6.  Part (d) guarantees that $H\cong S_3$.   Argue that $H$ must contain all of the elements of order 2 from $A_4$.
\item Explain why $A_4$ cannot have a subgroup of order 6. 
\end{enumerate}
\end{problem}

The next few problems explore the concept of quotient groups.  For additional information, check out Section 8.2 of the course notes. 

\blankline

Suppose $G$ is an arbitrary group and let $H\leq G$. Consider the set of left cosets of $H$ and define
\[
(aH)(bH):=(ab)H.
\]
The natural question to ask is whether this operation is well-defined.  That is, does the result of multiplying two left cosets depend on our choice of representatives?  More specifically, suppose $c\in aH$ and $d\in bH$.  Then $cH=aH$ and $dH=bH$.  According to the operation defined above, $(cH)(dH)=cdH$.  It better be the case that $cdH=abH$, otherwise the operation is not well-defined.

\begin{problem}
(4 points) Let $G$ be a group and let $H\leq G$.  Prove that left coset multiplication (as defined above) is well-defined if and only if $H\trianglelefteq G$. \emph{Note:} This is Theorem~8.29.
\end{problem}

\begin{problem}\label{prob:QuotientGrp}
(4 points) Let $G$ be a group and let $H\trianglelefteq G$.  Prove that the set of left cosets of $H$ in $G$ forms a group under left coset multiplication.  \emph{Note:} This is Theorem~8.30.
\end{problem}

The group from Problem~\ref{prob:QuotientGrp} is denoted by $G/H$, read ``$G$ mod $H$", and is referred to as the \textbf{quotient group} of $G$ by $H$.  If $G$ is a finite group, then $G/H$ is exactly the group that arises from ``gluing together" the colored blocks in a checkerboard-patterned group table.  It's also the group that we get after applying the quotient process to the Cayley diagram.  It's important to point out once more that this only works properly if $H$ is a normal subgroup.

\blankline

If $H\trianglelefteq G$, then $|G/H|$ is the number of left cosets of $H$ in $G$.  That is, $|G/H|=[G:H]$.  In particular, if $G$ is finite, then $|G/H|=|G|/|H|$.

\begin{problem}
(4 points) Let $G$ be a group.  Prove that if $G/Z(G)$ is cyclic, then $G$ is abelian.\footnote{Note that since the elements of $Z(G)$ commute with all the elements of $G$, the left and right cosets of $Z(G)$ will be equal, and hence $Z(G)$ is normal in $G$.}
\end{problem}

\begin{problem}
(4 points each) Prove any \textbf{two} of the following.
\begin{enumerate}[label=\rm{(\alph*)}]
\item Let $G$ be a group and let $H\trianglelefteq G$.  Prove that if $G$ is cyclic, then so is $G/H$. \emph{Note:} This is Theorem 8.39.
\item Prove that if $|G|=pq$, where $p$ and $q$ are primes (not necessarily distinct), then either $Z(G)=\{e\}$ or $G$ is abelian.
\item Suppose $G$ is a group of order $pq$ such that $p$ and $q$ are distinct primes (i.e, $p\neq q$). Prove that $G$ must have both an element of order $p$ and an element of order $q$.
\item Let $H$ be a normal subgroup of a group $G$.  Prove that if $g\in G$, then the order of $gH$ (in $G/H$) divides the order of $g$ (in $G$).
\end{enumerate}
\end{problem}

\begin{problem}
(2 points) Show that the converse of Problem 6(a) is not true by providing a specific counterexample. \emph{Note:} You should do this problem regardless of whether you chose to prove Problem 6(a).
\end{problem}

\end{document}