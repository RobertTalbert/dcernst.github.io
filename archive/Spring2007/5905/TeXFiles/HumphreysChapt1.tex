%\documentclass[12pt]{book}
\documentclass[12pt]{nauthesis}                       
  

\usepackage{amsmath, amsthm, amssymb}
\usepackage{epsfig}
\usepackage{amscd}
\usepackage{latexsym}
\usepackage{varioref}
\usepackage{subfigure}

\newtheorem{theorem}{Theorem}[section]
\newtheorem{corollary}[theorem]{Corollary}
\newtheorem{proposition}[theorem]{Proposition}
\newtheorem{lemma}[theorem]{Lemma}

\theoremstyle{definition} 
\newtheorem{definition}[theorem]{Definition}
\newtheorem{example}[theorem]{Example} 
\newtheorem{note}[theorem]{Note}

\newcommand{\Z}{\mathbb{Z}}
\newcommand{\R}{\mathbb{R}}
\newcommand{\C}{\mathbb{C}}
\newcommand{\D}{\mathbb{D}}
\newcommand{\Tor}{\textrm{Tor}}
\newcommand{\upw}{{^*_+w}}
\newcommand{\loww}{{^*_-w}}
\newcommand{\upz}{{^*_+z}}
\newcommand{\lowz}{{^*_-z}}
\renewcommand{\L}{\mathcal{L}}
\renewcommand{\(}{\big(}
\renewcommand{\)}{\big)}
\renewcommand{\c}{\cdot}%can put circ here


\begin{document}

\begin{flushleft}

\begin{chapter}{Finite Reflection Groups}

\begin{section}{Reflections}

\begin{definition}
Let $V$ be a real Euclidean space equipped with a positive-definite symmetric bilinear form $\langle\alpha, \beta\rangle$ (i.e., inner product; think dot product).  A {\it reflection} is a linear operator $s$ on $V$ s.t. 
\begin{enumerate}
\item[(i)] $s$ sends some nonzero vector $\alpha$ to $-\alpha$,
\item[(ii)] $s$ fixes pointwise the hyperplane $H_\alpha$ orthogonal to $\alpha$.
\end{enumerate}
\end{definition}

\medskip

Note that $\alpha$ is orthogonal to $\beta$ if $\langle\alpha,\beta\rangle=0$ and $H_\alpha$ is the subspace of vectors that are orthogonal to $\alpha$.  We will often write $s=s_\alpha$.  Note that $s_\alpha=s_{c\alpha}$ for nonzero $c \in \R$.

\medskip

\begin{proposition}
There is a formula for $s_\alpha$:
	$$s_\alpha(\beta)=\beta-\frac{2\langle\beta,\alpha\rangle}{\langle\alpha,\alpha\rangle}\alpha.$$
\end{proposition}

\begin{proof} To show that this is the correct formula, verify $s_\alpha$ that acts appropriately on $\alpha$ and any $\beta \in H_\alpha$.  Then note that $V=\alpha\R \oplus H_\alpha$.
\end{proof}

\medskip

One can show that $s_\alpha$ is an orthogonal transformation: 
	$$\langle s_\alpha(\gamma),s_\alpha(\beta)\rangle=\langle\gamma,\beta\rangle.$$
Note that ${s_\alpha}^2=1$, and so $s_\alpha$ has order 2 in the group $O(V)$ (the group of orthogonal transformations of $V$).

\medskip

\begin{definition}
A subgroup of $O(V)$ generated by reflections is called a {\it reflection group}.  If the group is finite it is called a {\it finite reflection group}.
\end{definition}

\medskip

The purpose of this chapter and the next is to classify and describe all finite reflection groups. 

\medskip

\begin{example}
Here are some basic examples.
\begin{enumerate}
\item[] ($I_2(m), m \geq 3)$:  Let $V=\R^2$ with the standard inner product.  Define $D_{2m}$ to be the {\it dihedral group} of order $2m$, consisting of the orthogonal transformations which preserve a regular $m$-sided polygon centered at the origin.  $D_{2m}$ consists of $m$ rotations (through multiples of $2\pi/m$) and $m$ reflections (about diagonals of polygon).   The rotations form a cyclic subgroup of index 2 (generated by a rotation through $2\pi/m$).

\medskip

Note that a rotation through $2\pi/m$ can be achieved as a product of 2 reflections relative to a pair of adjacent diagonals which meet at an angle of $\pi/m$.  Therefore, $D_{2m}$ is generated by reflections, and so it is a finite reflection group.  In fact, $D_{2m}$ is generated by any pair of reflections corresponding to adjacent diagonals.

\medskip

Let $H_\alpha$ and $H_\beta$ be the reflecting hyperplanes of any pair of adjacent diagonals.  For simplicity, assume that $H_\beta$ is the $x$-axis and $H_\alpha$ is $H_\beta$ after a positive rotation by $\pi/m$. Set $\alpha=(\sin\pi/m,-\cos\pi/m)$ and $\beta=(0,1)$.  Then $\alpha$ and $\beta$ form an obtuse angle of $\pi-\pi/m$.  To see that $s_\alpha s_\beta $ is a (counterclockwise) rotation through $2\pi/m$, determine the $2 \times 2$ matrices corresponding to $s_\alpha$ and $s_\beta$ and then multiply them.

\item[] $(A_{n-1}, n\geq 2)$:  Let $V=\R^n$ with standard inner product.  Consider the symmetric group $S_n$.  It can be thought of as a subgroup of $O(\R^n)$ in the following way.  Let $ \epsilon_1, \ldots, \epsilon_n$ be the standard basis for $\R^n$.  Let $S_n$ act on $\R^n$ by permuting the subscripts of the basis vectors.  Note that $(ij)$ acts as a reflection, sending $\epsilon_i-\epsilon_j$ to $\epsilon_j-\epsilon_i$ and fixing pointwise the orthogonal complement (which consists of all vectors of $\R^n$ with equal $i$th and $j$th components).  Since $S_n$ is generated by transpositions, it is a finite reflection group.

\item[] $(B_n, n \geq 2)$:  Again, let $V=\R^n$ with standard inner product.  Then $S_n$ acts on V as above.  Define additional reflections (called sign changes) that send an $\epsilon_i$ to its negative and fix all other basis vectors.  These new reflections generate a group of order $2^n$ isomorphic to $(\Z_2)^n$, which intersects $S_n$ trivially and is normalized by $S_n$ (conjugating the sign change $\epsilon_i \mapsto \epsilon_j$ yields another sign change).  So, the semidirect product of $S_n$ and the group of sign changes yields a finite reflection group of order $2^n n!$.  This group can also be thought of as the wreath product of $\Z_2$ with $S_n$.
\end{enumerate}
\end{example}

\medskip

Note that when $S_n$ acts on $\R^n$ in the way described above, it fixes pointwise the line spanned by $\epsilon_1 + \cdots + \epsilon_n$ (and these are the only points fixed) and leaves stable the orthogonal complement.  Therefore, $S_n$ also acts on an $(n-1)$-dimensional Euclidean space as a group generated by reflections, fixing no point except the origin.  This is the reason for the $n-1$ subscript in $A_{n-1}$.

\medskip

\begin{definition}
When a reflection group $W$ acts on $V$ with no nonzero fixed points, we say that $W$ is {\it essential} relative to $V$.
\end{definition}

\medskip

Any subgroup W of $O(V)$ stabilizes the orthogonal complement $V'$ of its space of fixed points and is essential relative to $V'$.

\end{section}

\newpage

\begin{section}{Roots}

See Brent's notes.  Here's a summary of the important stuff.

\medskip

Let $V$ be a Euclidean space.  Let $\Phi$ be a finite set of nonzero vectors in $V$.  Then $\Phi$ is a \emph{root system} if
\begin{enumerate}
  \item[(R1)] $\Phi \cap \R \alpha = \{\alpha, -\alpha\}$;
  \item[(R2)] $s_\alpha \Phi = \Phi$ for all $\alpha \in \Phi$.
\end{enumerate}

The elements of $\Phi$ are called \emph{roots}.  Let $W$ be the reflection group generated by all reflections $s_\alpha, \alpha \in \Phi$.  Then $W$ is in fact finite.  To see this, note that each $s_\alpha$ for $\alpha \in \Phi$ fixes pointwise the orthogonal complement of the subspace spanned by $\Phi$.  So, each $w \in W$ also fixes pointwise this complement.  Thus, only the identity in $W$ fixes all the elements of $\Phi$.  This implies that the natural homomorphism of $W$ into the symmetric group on $\Phi$ has trivial kernel, which forces $W$ to be finite.

\medskip

\begin{example}
\ 
\begin{enumerate}
  \item[(a)] (Type $I_2(4)$)  Let $\Phi=\{\pm(1,0), \pm(1,1), \pm(0,1), \pm(-1,1)\}$.  Then $\Phi$ is a root system and the associated finite reflection group has underlying group isomorphic to $D_8$.
  \item[(b)] (Type $A_n$)  Let $V=\R^{n+1}$ with standard basis $\epsilon_1, \dots, \epsilon_{n+1}$.  Let $\Phi=\{\epsilon_i -\epsilon_j : i \neq j \}$.  Then $\Phi$ is a root system and the associated finite reflection group has underlying group isomorphic to $S_{n+1}$.
\end{enumerate}
\end{example}

\begin{note}
We will study $W \subseteq O(V)$ in conjunction with a root system $\Phi \subset V$, subject only to (R1) and (R2).  The choice of $\Phi$ is somewhat flexible.  In might consist of unit vectors, or not.  The reflections $s_\alpha$ for $\alpha \in \Phi$ might or might not be known to exhaust all reflections in $W$.  The set $\Phi$ might span $V$, or not.  All that really matters is that (R1) and (R2) hold.
\end{note}

\end{section}


\begin{section}{Positive and Simple Systems}

See Brent's notes.  Here's a summary of the important stuff.

\medskip

Fix a root system $\Phi$ in the Euclidean space $V$, so that $W$ is the associated finite reflection group.  Also, fix a \emph{total ordering} of $V$ (total orderings exist; for our purposes think lexiographic ordering). We call $\Pi \subset \Phi$ a \emph{positive system} if it consists of all roots positive relative to the total ordering.  Since roots come in pairs $\{\alpha, -\alpha\}$, $\Phi$ is the disjoint union of $\Pi$ and $-\Pi$ (the \emph{negative system}).

\medskip 

A set $\Delta \subset \Phi$ is a \emph{simple system}  if 
\begin{enumerate}
  \item[(i)] $\Delta$ is a vector space basis for the $\R$-span of $\Phi$ in $V$; 
  \item[(ii)] Each $\alpha \in \Phi$ is a linear combination of $\Delta$ with coefficients all of the same sign.
\end{enumerate}

\begin{theorem}
\ 
\begin{enumerate}
  \item[(a)] If $\Delta$ is a simple system in $\Phi$, then there is a unique positive system containing $\Delta$.
  \item[(b)] Every positive system $\Pi$ in $\Phi$ contains a unique simple system; in particular, simple systems exist.
\end{enumerate}
\end{theorem}

\medskip

\begin{corollary}{\rm{(of proof)}}
$\langle \alpha, \beta \rangle \leq 0$ for all $\alpha \neq \beta$ in $\Delta$.
\end{corollary}

\begin{example}
\ 
\begin{enumerate}
  \item[(a)] (Type $I_2(4)$) $\Pi=\{(1,0), (1,1), (0,1), (1,-1)\}$ is a positive system and $\Delta=\{(0,1), (1,-1)\}$ is the unique simple system sitting inside of $\Pi$.
  \item[(b)] (Type $A_n$) $\Pi=\{\epsilon_i-\epsilon_j: i<j\}$ is a positive system and $\Delta=\{\epsilon_1-\epsilon_2, \epsilon_2-\epsilon_3, \dots, \epsilon_n-\epsilon_{n+1}\}$ is the unique simple system sitting inside of $\Pi$.
\end{enumerate}
\end{example}

\end{section}

\begin{section}{Conjugacy of Positive and Simple Systems}

See Brent's notes.  Here's a summary of the important stuff.

\medskip

\begin{proposition}
Let $\Delta$ be a simple system, contained in the positive system $\Pi$.  If $\alpha \in \Delta$, then $s_\alpha(\Pi\setminus \{\alpha\})=\Pi\setminus \{\alpha\}$.
\end{proposition}

\medskip

In other words, $\Pi$ and $s_\alpha\Pi$ (for $\alpha \in \Pi$) differ by only one root (namely $\alpha$ and $-\alpha$).  This result is often helpful in recognizing when a root is in fact equal to a given simple root $\alpha$: it characterizes $\alpha$ as the unique positive root made negative by $s_\alpha$.

\medskip

\begin{theorem}
Any two positive (respectively simple) systems in $\Phi$ are conjugate under $W$.
\end{theorem}

\begin{proof}
Let $\Phi$ be a root system and let $\Pi$ and $\Pi'$ be two positive systems (relative to some total orderings).  Then each contains exactly half the roots.  Let's induct on $r=\rm{Card}(\Pi \cap -\Pi')$.  If $r=0$, then $\Pi=\Pi'$.  Now, assume that $r>0$.  Then the simple system $\Delta$ in $\Pi$ cannot be fully contained in $\Pi'$.  Choose $\alpha \in \Delta$ with $\alpha \in -\Pi'$.  By the above proposition, $\rm{Card}(s_\alpha\Pi \cap -\Pi')=r-1$.  By induction (applied to $s_\alpha\Pi$ and $\Pi'$), there exists $w \in W$ such that $(ws_\alpha)\Pi=w(s_\alpha\Pi)=\Pi'$.  Slight modifications to this argument will give us the result for simple systems.
\end{proof}

\end{section}

\begin{section}{Generation by Simple Reflections}

Fix $\Delta \subset \Pi \subset \Phi$ and let $W$ be the corresponding finite reflection group.

\medskip

\begin{definition}
For $\alpha \in \Delta$, we call $s_\alpha$ a \emph{simple reflection}.
\end{definition}

\medskip

Note that each $\beta \in \Phi$ can be written uniquely
	$$\beta=\sum_{\alpha \in \Delta} c_\alpha \alpha.$$
We introduce the following definition to be used in proving an important theorem.

\medskip

\begin{definition}
Using the representation above for $\beta \in \Phi$, we define the \emph{height} of $\beta$ (relative to $\Delta$) via 
	$$\rm{ht}(\beta):=\sum_{\alpha \in \Delta} c_\alpha.$$
\end{definition}

\medskip

For example, $\rm{ht}(\beta)=1$ if $\beta \in \Delta$.

\medskip
	
\begin{theorem}
For a fixed simple system $\Delta$, $W$ is generated by simple reflections.
\end{theorem}

\begin{proof}
Let $W'$ be the subgroup generated by the simple reflections.  We will show that $W'=W$.  There are 3 main steps.
\begin{enumerate}
  \item[(1)] If $\beta \in \Pi$, consider $W'\beta \cap \Pi$.  This set contains $\beta$, and so it is nonempty.  Choose from this set and element $\gamma$ of smallest height.  We will show that $\gamma \in \Delta$.  Suppose
  	$$\gamma=\sum_{\alpha \in \Delta} c_\alpha \alpha.$$
Note that 
	$$0<\langle \gamma, \gamma \rangle=\sum c_\alpha \langle \gamma, \alpha \rangle,$$
which forces $\langle \gamma, \alpha \rangle >0$ for some $\alpha \in \Delta$.  Now, if $\gamma=\alpha$, then $\gamma$ is a simple root.  Suppose that $\gamma$ is not $\alpha$.  Consider the root $s_\alpha \gamma$, which is positive by the proposition in 1.4.  Since $s_\alpha \gamma$ is obtained from $\gamma$ by subtracting a positive multiple of $\alpha$, we have $\rm{ht}(s_\alpha \gamma)< \rm{ht}(\gamma)$.  But $s_\alpha \gamma \in W'\beta$ (since $s_\alpha \in W'$), contradicting the original choice of $\gamma$.  Therefore, $\gamma=\alpha$, and so $\gamma$ is simple.  That is, the $W'$-orbit of any positive root $\beta$ meets $\Delta$, so that $\Pi \subseteq W'\Delta$.
  \item[(2)] Now, we show that $W'\Delta=\Phi$. If $\beta$ is a negative root, then $-\beta \in \Pi$.  By step (1), there exists $w \in W'$ and $\alpha \in \Delta$ such that $-\beta=w\alpha$.  This implies that 
  	$$(ws_\alpha)\alpha=w(s_\alpha \alpha)=w(-\alpha)=-w\alpha=\beta.$$
But $ws_\alpha \in W'$ (since $w$ and $s_\alpha$ are).  So, $-\Pi \subseteq W'\Delta$.  Therefore, $\Phi=W'\Delta$.
  \item[(3)] Finally, take any generator $s_\beta$ of $W$ ($\beta$ any element of $\Phi$).  By step (2), we can write $\beta=w\alpha$ for some $w \in W'$ and some $\alpha \in \Delta$.  By a fact that Brent mentioned in 1.2, $s_\beta=ws_\alpha w^{-1} \in W'$.  This proves that $W=W'$.
\end{enumerate}
\end{proof}

\begin{corollary}{\rm{(of proof)}}
Given $\Delta$, for every $\beta \in \Phi$ there exists $w \in W$ such that $w\beta \in \Delta$.
\end{corollary}

\end{section}

\end{chapter}

\end{flushleft}

\end{document} 