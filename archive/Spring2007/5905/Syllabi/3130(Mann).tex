% Course syllabus for Math 3130-300 Summer 2006
% LaTeX by Allen Mann

\documentclass[11pt]{article}
\usepackage{graphicx}
\usepackage{amssymb}

\textwidth = 6.5 in
\textheight = 9.5 in
\oddsidemargin = 0.0 in
\evensidemargin = 0.0 in
\topmargin = 0.0 in
\headheight = -0.25 in
\headsep = 0.0 in
\parskip = 0.2in
\parindent = 0.0in

%Allen's Macro's
\newcommand{\AM}{\textsc{a.m.}}
\newcommand{\PM}{\textsc{p.m.}}
\newcommand{\book}[1]{\textit{#1}} 
\newcommand{\url}[1]{\texttt{#1}} 

\begin{document}

\begin{center}
\framebox{\parbox{4 in}{
\begin{center}
\textbf{\large MATH 3130-300} \\[6 pt]
\textbf{\large Introduction to Linear Algebra} \\[10 pt]
\textbf{Summer 2006}
\end{center}
}}
\end{center}


\begin{tabbing}
\textit{Office Hours:} \quad   \= Allen.Mann@Colorado.EDU   \kill
\textit{Instructor:}		\> Allen Mann \\
\textit{Office:}			\> Mathematics 362 \\
\textit{E-mail:}			\> \url{Allen.Mann@Colorado.EDU} \\
\textit{Web:}			\> \url{http://math.colorado.edu/\~\,$\!$almann/math3130}
\end{tabbing}

\vspace{-0.2 in}

\begin{description}

\item[Lecture:] MTWRF 9:15--10:15 \AM\ in ECCR 116.

\item[Office Hours:] By appointment only.

\item[Textbook: ]
\book{Linear Algebra and Its Applications} (3rd edition), by David C. Lay. Pearson-Addison-Wesley, 2006.

\item[Course Description:]
This course examines basic properties of systems of linear equations, vector spaces, linear independence, dimension, linear transformations, matrices, determinants, eigenvalues, and eigenvectors. Students may not receive credit for both MATH 3130 and APPM 3310.

\item[Prerequisites:] MATH 2400 Analytic Geometry and Calculus 3. MATH 3000 Introduction to Abstract Mathematics is helpful, but not required.

\item[Grading: ]
\begin{tabbing}
\hspace{0.5 in}    \= Problems of the Week \quad  \=    \kill
\> Daily Homework  \> 10\%    \\
\> Weekly Homework  \> 10\%    \\
\> Quizzes  \> 10\%    \\
\> Exam 1  \> 20\%    \\
\> Exam 2  \> 20\% \\
\> Final Exam \> 30\% \\ 
\end{tabbing}
\vspace{-0.18 in}
The course will not be curved. I have standards that you must meet.

\item[Homework \& Quizzes: ]
Daily homework assignments will be due at the beginning of the following class and will be graded on completion only. Weekly homework assignments and quizzes will be graded for correctness. Students are encouraged to work together on homework, but you should write up your solutions on your own.

\item[Calculators: ] Calculators will not be allowed on quizzes or exams.

\item[Tutoring: ] If you require additional help, the math department office (Mathematics 260) has a list of certified tutors and the rates they charge.

\item[Final Exam: ]
Friday, July 28, 9:15--10:15 \AM

\pagebreak

\item[Students with Disabilities: ]
Students with disabilities who qualify for academic accommodations must provide a letter from Disability Services and discuss specific needs with me, preferably during the first two weeks of class.  Disability Services determines accommodations based on documented disabilities. Disability Services in located in Willard 322, (303) 492-8671, \\ \url{http://www.colorado.edu/sacs/disabilityservices}.

\item[Religious Obligations: ]
Students who have a religious obligation that conflicts with one of the scheduled exams, assignments, or other required attendance should notify me at least two weeks in advance of the conflict to request special accomodation. Please refer to the University's policy on religious obligations at \url{http://www.colorado.edu/policies}.

\item[Classroom Behavior: ]
See \url{http://www.colorado.edu/policies/classbehavior.html}.

\item[Honor Code: ] 
Details of the Student Honor Code system can be found at \\
\url{http://www.colorado.edu/academics/honorcode} \\
\url{http://www.colorado.edu/policies/honor.html} \\
\url{http://www.colorado.edu/policies/acadinteg.html}

\item[Cheating: ]
Cheating is defined as using unauthorized materials or receiving unauthorized assistance during an examination or other academic exercise. Examples of cheating include: copying the work of another student during an examination or other academic exercise (includes computer programming), or permitting another student to copy one�s work; taking an examination for another student or allowing another student to take one�s examination; possessing unauthorized notes, study sheets, examinations, or other materials during an examination or other academic exercise; collaborating with another student during an academic exercise without the instructor�s consent; and/or falsifying examination results.

\item[Plagiarism: ]
Plagiarism is defined as the use of another�s ideas or words without appropriate acknowledgment. Examples of plagiarism include: failing to use quotation marks when directly quoting from a source; failing to document distinctive ideas from a source; fabricating or inventing sources; and copying information from computer-based sources, e.g., the Internet.

\item[Unauthorized Possession or Disposition of Academic Materials: ]
Unauthorized possession or disposition of academic materials may include: selling or purchasing examinations, papers, reports or other academic work; taking another student�s academic work without permission; possessing examinations, papers, reports, or other assignments not released by an instructor; and/or submitting the same paper for multiple classes without advance instructor authorization and approval.

\end{description}

\end{document}